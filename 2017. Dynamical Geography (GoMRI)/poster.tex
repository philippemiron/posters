%=================================
%====template for LATEX poster==== 
%=================================

\documentclass[final]{beamer}
%\usepackage[orientation=portrait,size=a0,
%            scale=1.25         % font scale factor
%        ]{beamerposter}

% modify for 36x48in poster
\usepackage{beamerposter}
\setlength{\paperwidth}{48in}
\setlength{\paperheight}{36in}

\geometry{
  hmargin=2.5cm, % little modification of margins
}

%
\usepackage[utf8]{inputenc}

\linespread{1.15}
%
%==The poster style============================================================
\usetheme{sharelatex}

%==Title, date and authors of the poster=======================================
\title
[Gulf of Mexico Oil Spill \& Ecosystem Science Conference, February 5--10 2017,
New Orleans, USA] % Conference
{ % Poster title
A Dynamical Geography of the Gulf of Mexico
}

\author{ % Authors
P. Miron\inst{1}, %, Author Two\inst{2}, Author Three\inst{2,3}
F. J. Beron-Vera\inst{1},
M. J. Olascoaga\inst{1},
 Paula P\'erez-Brunius\inst{2}, 
 Julio Sheinbaum\inst{2} and 
 Gary Froyland\inst{3}
}
\institute
[Very Large University] % General University
{
\inst{1} Rosenstiel School of Marine and Atmospheric Science, University of
Miami, USA
\\[0.3ex]
\inst{2} CICESE, Ensenada, Mexico
\\[0.3ex]
\inst{3} University of New South Wales, Sydney, Australia
}
\date{\today}

% other useful packages
\usepackage[super,sort&compress]{natbib}
\usepackage{siunitx}
\usepackage{tikz}
\newcommand{\PF}{\mathcal{P}}
\newcommand{\ia}{\textit{a}}
\newcommand{\ib}{\textit{b}}
\newcommand{\ic}{\textit{c}}
\newcommand{\id}{\textit{d}}
\newcommand{\ie}{\textit{e}}
\newcommand{\gom}{GoM}
\let\vaccent=\v 
\renewcommand{\v}[1]{\ensuremath{\mathbf{#1}}} 
\newcommand{\minus}{\scalebox{0.5}[1.0]{$-$}}

\graphicspath{{"figures/"}}

\begin{document}
\begin{frame}[t]
\begin{multicols}{3}

% The poster content
\section{Introduction}
Density dispersion in fluid flow is difficult to predict as it involves many
mechanisms affecting a wide range of time and length scales. In the present
study, we use the available drifter trajectories database (from 1994--2016) to
extract almost-invariant regions and predict transport in the Gulf of Mexico
(\gom). A total of 3207 drifter trajectories from several sources were
considered and all the data points are presented in Fig.~\ref{fig:gom}. These
drifters were deployed over many years and their design differs from experiment
to experiment, so some variations in their Lagrangian properties can be
expected.  For the purposes of this work, these variations are ignored as the
main goal is the analysis of the average dynamics of the \gom.
\begin{figure}
\centering
\includegraphics[width=0.9\columnwidth]{figures/fig03}
\caption{All data points of drifter trajectories are plotted on the left panel
and the drifters density in each bin (from 1-4266 data points per bin with an
average of 246) is presented on the right panel.}
\label{fig:gom}
\end{figure}

\section{Method}
The eigenvector method\citep{froyland2014well} employed is rooted in
Markov-chain concepts that have led to the possibility of approximating
invariant sets in dynamical systems using short-run
trajectories\citep{dellnitz1997almost}. The dynamical system of interest is
that governing the motion of the fluid particles, which are described by
satellite-tracked drifters on the ocean surface.

Let $X$ be a closed 2D flow domain and denote by $T(x)$ the end point of a
trajectory starting at $x \in X$ after some short time. A discretization of the
dynamics can then be attained using Ulam's method\citep{ulam1960,Froyland-01}
by dividing the domain $X$ into $N$ boxes $\left(B_1,\cdots,B_N\right)$. The
probability of going from a box  $B_i$ to a box $B_j$ under one application of
$T$ is (approximately) equal to
\begin{equation*}
\PF_{ij} \approx \frac{\#\lbrace d: d \in B_i \text{ and } T(d) \in
B_j\rbrace}{\#\lbrace d \in B_i\rbrace} \text{ with } d \text{ the individual
drifter.}
\end{equation*}

The transition matrix $\PF$ defines a Markov Chain of the dynamics, which is a
stochastic model describing a sequence of possible events in which the
probability of each event depends only on the current state. From an initial
density $f_0$, future distribution can be approximate by $\mathbf{f}_k =
\mathbf{f}_0 \PF^k$.

The left eigenvector ($\lambda L = L \PF$) with $\lambda = 1$ describes the
limiting distribution of the system while the corresponding right eigenvector
is supported on the basin of attraction of the distribution.  Furthermore,
eigenvectors associated with eigenvalues smaller than one allow the extraction
of almost-invariant sets, which are relevant in dynamical systems governing
motion that exhibits transient behavior.

\subsection{Example using a reduced Markov chain}
\begin{figure}[!ht]
  \centering
  \includegraphics[width=0.8\columnwidth]{figures/fig01}
\caption{The probability of moving from one state to another and if a state or
group of states belongs to a communicating class (CC) or an absorbing closed
communicating class (ACCC) are identified.}
  \label{fig:ccc}
\end{figure}
On the associated transition matrix, there is one line for each state of the
system and the values in each column represents the probability of going from
state $i$ to state $j$.

\vspace{0.5cm}
\columnbreak
From left eigenvectors $L_{1-2}$, one can identify both ACCC $\lbrace A,
B\rbrace$ and  $\lbrace E \rbrace$ while their respective basins of attraction
are identified by the corresponding right eigenvectors $R_{1-2}$.

\begin{equation*}
\PF = 
  \begin{tabular}{c|ccccc}
    & A & B & C & D & E\\\hline
  A & 0.8 & 0.2 & 0 & 0 & 0\\
  B & 0.7 & 0.3 & 0 & 0 & 0\\
  C & 0 & 0.2 & 0.8 & 0 & 0\\
  D & 0 & 0 & 0 & 0.7 & 0.3\\
  E & 0 & 0 & 0 & 0 & 1.0
  \end{tabular}\quad
  L_1^T = 
  \begin{pmatrix}
  0.83\\
  0.55\\
  0\\
  0\\
  0
  \end{pmatrix}\quad
  R_1 = 
  \begin{pmatrix}
  1\\
  1\\
  1\\
  0\\
  0
  \end{pmatrix}\quad
  L_2^T = 
  \begin{pmatrix}
  0\\
  0\\
  0\\
  0\\
  1
  \end{pmatrix}\quad
  R_2 = 
  \begin{pmatrix}
  0\\
  0\\
  0\\
  1\\
  1
  \end{pmatrix}
\end{equation*}

\section{Results and discussions}
The eigenvector method is now applied to the transition matrix $\PF$
constructed from the drifter trajectories using a 2 day transition time. The
left eigenvector associated with the largest eigenvalue less than one
identifies the main attractor (Fig.~\ref{fig:l_eig}\ia) on the right part of
the domain, communicating with the Atlantic Ocean. The associated right
eigenvector (Fig.~\ref{fig:r_eig}\ia) covered the whole domain as stated by the
Perron--Frobenius theory, except from the small isolated close communicating
classes (CCC). This states that with the global circulation of the Gulf of
Mexico, any density is diluted, trapped by the Loop Current into the Florida
Strait and evacuated to the Atlantic Ocean.

\begin{figure}
\centering
\includegraphics[width=\columnwidth]{figures/fig06}
\caption{Left eigenvectors showing the principal attractive regions.}
\label{fig:l_eig}
\end{figure}

Fig.~\ref{fig:l_eig}\ib\ identifies the main attractor inside the \gom. It is
located on the western boundary of the \gom\ on the Tamaulipas (Mexico) Gulf
Coast while its basin of attraction (Fig.~\ref{fig:r_eig}\ib) splits the \gom\
in two at the west of the Yucat\'{a}n Channel.
 
Fig.~\ref{fig:l_eig}\ic\ highlights an attractive region on the West Florida
Shelf while the next eigenvectors (bottom of Fig.~\ref{fig:l_eig}) emphasize on
the presence of 4 different attractors; on  the south shore of Cuba, the SW
coast of the \gom, at the northern part of the West Florida Shelf and on the NW
shore of the \gom\ on the Texas Gulf Coast.
\begin{figure}
\centering
\includegraphics[width=\columnwidth]{figures/fig07}
\caption{Right eigenvectors showing the corresponding basins of attraction of
the attractive regions in Fig.~\ref{fig:l_eig}.}
\label{fig:r_eig}
\end{figure}

\vspace{0.5cm}
\columnbreak
\subsection{Dynamical geography}
The dynamical geography is a partition of the surface of the \gom\ developed by
combining the basins of attraction presented in Fig.~\ref{fig:r_eig}. On the
left, the \gom\ is split into two regions, one connecting the Caribbean Sea to
the Atlantic while the other is isolated on the western side. The separation of
the \gom\ fit well with the boundary of the Loop Current, identified in diverse
papers\citep{maze2015historical}. On the right, five weakly interacting coastal
basins are extracted by thresholding the support area of the right
eigenvectors. Each coastal basin is characterized by a distinct flow dynamics
and coherence for important time scale. Those regions are consistent with
observations of slow dynamics around the Mississippi Delta, of more intensively
mixing region around the Florida Coast and of transient eddy structures south
of the Island of Cuba.
\begin{figure}
\centering
\includegraphics[width=0.8\columnwidth]{figures/fig09}
\end{figure}

The transition matrix $\PF$ can also be used to predict the dispersion of
passive tracers such as oil spills (Fig.~\ref{fig:oil}). Each row shows the
evolution of  two distinct oil spills, one occurring in a coastal basin and the
other one further away from the coast. While the top row density stays mostly
inside the basin during all the \SI{36}{d} period, the bottom row evolution
presents a wider spread due to the increased mixing and transport further away
from the coast.

\begin{figure}
\centering
\includegraphics[width=0.8\columnwidth]{figures/fig_oil}
\caption{Each row presents the evolution of two different oil spills inside the
\gom. The left column shows the initial location of the density while the 2nd
and 3rd columns present the evolution after 16 and 36 days.}
\label{fig:oil}
\end{figure}

\section{Conclusions}
We construct a Markov-chain representation of the surface-ocean Lagrangian
dynamics in the region spanned by the \gom\ and the northern Caribbean Sea
using a very large collection of drifter trajectories. From the analysis of the
eigenvalues and eigenvectors of the transition matrix associated with the
chain, we identify almost-invariant attracting sets and their basins of
attraction. With this information, we decompose the GoM's geography into weakly
dynamically interacting provinces in both forward and backward time, which has
implications for the connectivity of passive (Lagrangian) and potentially also
non-passive (e.g., chemically reacting, biologically active) tracers in the
\gom.

% copy bbl data for archive
\bibliographystyle{abbrvnat}
\begingroup
\renewcommand{\section}[2]{}%
\begin{thebibliography}{5}

\bibitem[Dellnitz and Junge(1997)]{dellnitz1997almost}
M.~Dellnitz and O.~Junge.
\newblock Almost invariant sets in chua's circuit.
\newblock \emph{International Journal of Bifurcation and Chaos}, 7\penalty0
  (11):\penalty0 2475--2485, 1997.

\bibitem[Froyland(2001)]{Froyland-01}
G.~Froyland.
\newblock Extracting dynamical behaviour via markov models.
\newblock \emph{Nonlinear Dynamics and Statistics:
  Proceedings of the Newton Institute, Cambridge, 1998}, pages 283--324.
  Birkhauser, 2001.

\bibitem[Froyland et~al.(2014)Froyland, Stuart, and van
  Sebille]{froyland2014well}
G.~Froyland, R.~M. Stuart, and E.~van Sebille.
\newblock How well-connected is the surface of the global ocean?
\newblock \emph{Chaos: An Interdisciplinary Journal of Nonlinear Science},
  24\penalty0 (3):\penalty0 033126, 2014.

\bibitem[Maze et~al.(2015)Maze, Olascoaga, and Brand]{maze2015historical}
G.~Maze, M.~Olascoaga, and L.~Brand.
\newblock Historical analysis of environmental conditions during florida red
  tide.
\newblock \emph{Harmful Algae}, 50:\penalty0 1--7, 2015.

\bibitem[Ulam(1960)]{ulam1960}
S.~M. Ulam.
\newblock \emph{A collection of mathematical problems}, volume~8.
\newblock Interscience Publishers, 1960.
\end{thebibliography}
\endgroup
\vspace{0.5cm}
\end{multicols}
\end{frame}
\end{document}